%!TEX root = paper.tex

\subsection{Discovery Mechanisms} % (fold)
\label{ssec:discovery}

% Due to the diversity of building blocks available, there is a necessity for different kinds of discovery or recommendation mechanisms, depending on the level of composition: screen-flow or screen.

The main goal of the discovery process is to aid the user in finding suitable building blocks to complement the ones they are already using. E.g., on the screen-flow level, this would mean to suggest additional screens to make existing screens within a screen-flow reachable, and the screen-flow therefore executable.
The platform offers two mechanisms to find and recommend screens (or other building blocks) stored in the catalogue: a simple discovery based on pre- and postconditions, and a multi-step discovery or planning algorithm. 
Before being presented to the user, the results are being ranked, as discussed in Sect.~\ref{sssec:ranking}. 

\subsubsection{Simple Discovery Based on Pre- and Post-Conditions}
\label{sssec:simple_discovery}

In this simple approach, the platform will assist the process by recommending all building blocks which will satisfy currently unfulfilled pre-conditions. 
The pre-conditions of all the unreachable building blocks are collected as a graph pattern, which is then matched against the post-conditions of all available building blocks. 
In the following scenario, there are two 
screens: \emph{s1} and \emph{s2}. \emph{s1} has as a pre-condition: \emph{``there exists a search criteria''}, and as a 
post-condition: \emph{``there exists a item''}. \emph{s2} has just a pre-condition stating: \emph{``there exists a search criteria''}.

\begin{listing}
\begin{verbatim}
:G1 { :s1 a fgo:Screen .      
      :s1 fgo:hasPrecondition c1 .
      :s1 fgo:hasPostcondition c2 .
      :c1 fgo:hasPattern GC1 .
      :c2 fgo:hasPattern GC2 .
      :s2 a fgo:Screen .      
      :s2 fgo:hasPrecondition c3 .
      :c3 fgo:hasPattern GC3 }
:GC1 { _:x a amazon:SearchCriteria }
:GC2 { _:x a amazon:Item }
:GC3 { _:x a amazon:SearchCriteria }
\end{verbatim}
\label{lis:discovery_example}
\end{listing}

The algorithm will construct a SPARQL query to retrieve building blocks satisfying the pre-conditions \emph{c1} and \emph{c3}. 
The query, although simplified for the sake of clarity, would look something like:

\begin{listing}
\begin{verbatim}
PREFIX rdf: <http://www.w3.org/1999/02/22-rdf-syntax-ns#>
PREFIX fgo: <http://purl.oclc.org/fast/ontology/gadget#>
PREFIX amazon:<http://aws.amazon.com/AWSECommerceService#>

SELECT DISTINCT ?bb 
WHERE { 
  ?bb rdf:type fgo:Screen . 
  {
    {
      ?bb fgo:hasPostCondition ?c .
      ?c fgo:hasPattern ?p .
      GRAPH ?p { ?x rdf:type amazon:SearchCriteria } 
    }
    UNION
    {
      ?bb fgo:hasPostCondition ?c .
      ?c fgo:hasPattern ?p .
      GRAPH ?p { ?x rdf:type amazon:Item } 
    }
  }
  FILTER (?bb != <http://fast.org/screens/S1>) 
  FILTER (?bb != <http://fast.org/screens/S2>) 
}
\end{verbatim}
\label{lis:sparql_find}
\end{listing}

However, in the example presented, it is straight-forward to spot that the pre-condition \emph{c3} would be
satisfied whether the screen \emph{s1} was reachable, being unnecessary to query the recommender algorithm 
to retrieve screens satisfying this pre-condition, hence these pre-conditions are removed for the construction
of the query reducing the complexity of the problem, hence improving the overall performance of the algorithm.
Moreover, it leads to find better results for the user, since the focus is given to make satisfy the 
pre-conditions which could not be satisfied by any of the elements of the current composition.

\subsubsection{Enhanced discovery: search tree planning}
\label{sssec:planning}

In artificial intelligence, the term \emph{planning} originally meant a search for a sequence of logical operators or actions
that transform an initial world state into a desired goal state. Presently, it also includes many decision-theoretic ideas,
imperfect state information, and game-theoretic equilibria.

This paper applies the concept of planning to the design of building blocks. Back to the screen-flow design, the goal
would be to make the screen-flow executable; the initial state would be a certain screen which is not yet reachable, 
and the plans would be sets of screens which are reachable by themselves, and accomplish the goal. 
The algorithm has been influenced by the ideas behing the heuristic search. For a certain state, i.e. the initial 
state which contains the screen to make reachable, a large tree of possible continuations is considered, in 
fact any screen would fit. Those screens which post-conditions do not satisfy in any form the unsatisfied 
pre-conditions are discarded, reducing the branches of the tree. A branch stops growing when a screens is
already reachable (i.e. it has no pre-conditions) becoming a leaf of the tree. Once there are not screens 
added in a certain step, the algorithm stops and discards the incomplete branches, in other words, 
those branches where the leaf is not reachable.

It is worth pointing out some of the tree structure is pre-computed beforehand to speed up the querying process at runtime. 
Any time a building block is inserted into the catalogue, the algorithm is executed following two approaches: 
\emph{forward search} and \emph{backward search}. The forward search approach finds the building blocks whose 
pre-conditions will be satisfied by the post-conditions of the new building block while the backward search 
finds the building blocks whose post-conditions will satisfy the pre-conditions of the recently
created building block. This data is stored and used for the tree or plan builder algorithm, without
needing to check the compatibility of pre- and post-conditions for every single building block at every 
request. 

\subsubsection{Results ranking}
\label{sssec:ranking}

Previous sections presented different mechanisms to search or discover building blocks which may satisfy the user needs.
The results ensure compatibility based on the functional specification of the building blocks, however they are not
ranked, and their quality is not measured. This section explains the ranking techniques applied for the different
discovery mechanisms.

The ranking algorithm for~\ref{sssec:simple_discovery} applies the following rules:
\begin{enumerate}
 \item it gives a higher position to those building blocks which satisfy the highest number of pre-conditions;
 \item it prioritise building blocks created by the same user who is querying;
 \item it adjust the rank by using the ratings given to the building blocks, and their popularity in terms of usage statistics;
 \item it weights the results according to non-functional features such as availability. This is only applied for what we call
``web service wrapper'', and it is calculated periodically by invoking the wrapped web services generating an up-time rate.
\end{enumerate}

For the planning case, the objective is not only to produce a plan but also to satisfy user-specified preferences, or what is
known as \emph{preference-based planning}. The ranking algorithm:
\begin{enumerate}
 \item minimizes the size of the plans, after removing the elements of the plan which are already in the canvas, so it gives
priority to the elements the user has already inserted,
 \item adjust the rank by using the rules 2, 3 and 4 used from the ranking algorithm of simple discovery based on 
pre- and post-conditions.
\end{enumerate}
