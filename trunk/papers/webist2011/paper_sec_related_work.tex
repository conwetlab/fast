%!TEX root = paper.tex

\section{Related Work}
\label{sec:related_work}

Web services have been around for a long time.
One of the most important claims about their benefits has been (syntactic) interoperability between third-party systems and applications based on different platforms / programming languages. 
System integration, within a company or between systems from different enterprises, became easier with the adoption of web service standard technologies such as WSDL. 
While many integrated development environments (IDEs) can deal with WSDL to facilitate the integration task, trained developers are still required for this.
% however it is still required that developers read and understand their specifications, and implement part of the logic programmatically. 
As a step forward, there are several tools which facilitate the interaction with data sources and services on the Web. Yahoo! Pipes, Apatar, JackBe Presto and NetVibes, among others provide a set of modules to access different kind of data sources, such as RSS feeds, a given web page (HTML code), Flickr images, databases, and even powerful enterprise systems such as Salesforce CRM or Goldmine CRM. 
However, none of these solutions facilitate end-users to build their own applications allowing the interaction with web services created by third-party providers. The solutions found are mainly data-oriented (RSS feeds, databases, raw text). Several tools permits some sort of (web) service integration, but are meant to be used within an enterprise level by savvy business users or developers. The solution presented in this paper leverages the possibility of integrating RESTful or SOAP-based web services inside browser-based applications (i.e., widgets or gadgets) by providing a platform to create, publish and select web services wrappers.

In the context of publishing and discovering Web services, service providers have well-known and widely used technologies to accomplish the task of publication, such as the Universal Description, Discovery and Integration (UDDI)~\cite{uddi2004}. UDDI serves as a centralised repository of WSDL documents. A similar concept is iServe~\cite{pedrinaci_ores2010}. This platform aims to publish web services as what they called Linked Services --- linked data describing services ---, storing web service definitions as semantic annotations, so that other semantically aware applications may take advantage of it. However, the platform does not handle the step from definition to consumption of the services.

% A different approach for web service discovery is the Web Services Dynamic Discovery (WS-Discovery) specification~\cite{beatty2005}. The core of this approach is a multicast discovery protocol. Service providers and consumers listen to each other for new services specifications within a network, so there is no need of a centralised registry. As a drawback, WS-Discovery do not support Internet-scale discovery, making it useless for our purposes.
