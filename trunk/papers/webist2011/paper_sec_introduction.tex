%!TEX root = paper.tex

\section{Introduction}
\label{sec:introduction}

Business-to-business (B2B) integration is still a significant challenge, often requiring extensive efforts in terms of different aspects and technologies for protocols, architectures or security.
While the adoption of open standards 
such as RosettaNet, ebXML, the Web Service Description Language (WSDL), Universal Description, Discovery and Integration (UDDI) or the Simple Object Access Protocol (SOAP)
have reduced the complexity of integrating business applications between different companies and partners,
% In this sense, Web services and XML standards such as RosettaNet, ebXML or OAG have been a boon to the world of B2B. By web services standards we are referring to the following open standards: Web Services Description Language (WSDL --- to describe), Universal Description, Discovery and Integration (UDDI --- to advertise and syndicate), Simple Object Access Protocol (SOAP --- to communicate) and Web Services Flow Language (WSFL --- to define work flows). 
% A common scenario  in their usage is a company using WSDL to describe its web services, publishing them through a repository using UDDI, while these web services use SOAP-based messages to achieve dynamic integration between different, disparate applications.
and offered some advantages in business-to-consumer (B2C) integration as well, the interaction and consumption of web services still requires programming skills and a deep understanding of the  technology, which poses an obstacle for an end-user with limited knowledge of the matter. % (in the sense as defined in~\cite{fuchsloch2010}). 
Regarding the selection of the right service for the right task --- a crucial requirement for the dynamic use of web services --- most publishing platforms are syntax-based, %leading to poor precision and recall in finding the correct service, 
making it difficult to navigate through a large number of web services~\cite{pilioura_acm2009}, 
preventing end-users from performing these tasks. 
Solutions such as Semantic Web Services (SWS) promised many advantages in this respect.
% discovery, composition and consumption of web services, independently of the provider's platform, location, service implementation or data format. 
However, as of yet they have not been widely adopted, 
possibly because the perceived potential benefits did not justify the additional investments~\cite{shi2007}.
% In a similar vein as B2C, governments are currently opening their data to the public as linked data. 
% So far, the efforts are restricted to data only, so that users are left alone in finding applications of that data. 
% A natural further development in the public sector would be that their services for administration tasks and single-window applications will be opened as well. Hence, governments will benefit from the research done around these concepts and technologies.

The motivation of our work has been strongly influenced by the end-users' needs. We are targeting users which are non-skilled in programming and software development, empowering them with a platform to select and consume web services in a straight-forward manner.
Rather than publishing services directly in our platform, we leave existing third-party services untouched and instead integrate them through service wrappers. The products resulting from the wrapping process are two artifacts: a specific definition of the web service to be used by this tool, and a ready-to-use piece of code with the proper functions to invoke the web service.

% The remainder of this paper is organised as follows. Section~\ref{sec:related_work} covers the state of the art in web services publication, discovery and wrapping/consumption. In Section~\ref{sec:wrapping_web_services} we present our approach for web services wrapping, available possibilities and limitations. Next, Section~\ref{sec:discovery} describes our platform for publishing web services and empowering end-users to discover services. Section~\ref{sec:use_case} gives the reader a clearer use case and how this work is integrated into the FAST platform, and finally, we present our main conclusions and highlight future lines of research in Section~\ref{sec:conclusions}.

