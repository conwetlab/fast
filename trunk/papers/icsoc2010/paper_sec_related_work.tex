
\section{Related work}
\label{sec:related_work}

Web services have been around for a long time since they first appeared. One of their most important claims has been (syntactic) interoperability between third-party systems and applications based on different platforms / programming languages. System integration, within a company or between systems from different enterprises became easier with the adoption of the web services standard technologies, being the Web Service Description or WSDL the most extended standard used for the definition of web services. In fact, there are many integrated development environments (IDEs) that can deal with WSDL, and create a plain vanilla proxy class for the desired platform / programming language, getting a more ``developer-friendly'' API built around proxy classes, instead of having to know all request parameter names and possible values for the different calls to the Web service. Even though, in an integration of two or more systems scenario, developers must read and understand their specifications in order to accomplish the task, in a programmatically-fashion way. Another advantage of the use of WSDL is validation. Having the request messages (methods and parameters) and response messages (return parameters / structures) built upon XML schemas, a Web service engine can validate the message conforming to a schema. However, as it has been said before, the use of these languages and tools requires a qualified developer. In a inter-company system integration scenario these technologies are of high value, but when a company opens its services to the public, it is clear that the common end-user has been left off-side.

There are several tools out there situated a step forward to facilitate the interaction with data sources and services from around the Web. Yahoo! Pipes provides a set of modules to access different kind of data sources, such as RSS feeds, a given web page (HTML code), Flickr images, Google base or the Yahoo! search engine, among others. These modules are developed by the widget platform itself, they do not allow users to build their own modules, and it lacks of a way to interact with REST or SOAP web service. Another tool worth to mention is Apatar, an enterprise-oriented data integration and ETL tool. It allows the interaction with databases (MySQL, PostgreSql, Oracle), powerful enterprise systems such as Salesforce CRM, SugarCRM and Goldmine CRM, and access to data sources such as XML or text files. It enables non-developers to design and perform data transformations in a visual way through its graphical interface; however it misses support for Web services. Yet another tool, in the same direction as Apatar, could be JackBe Presto, defined as an enterprise-ready mashup solution. Presto permits to integrate and extract data from a great variety of data sources such as RSS and ATOM feeds or XML-based sources, and empower the user to create connectors to use services from HP SOA Systinet and any Oracle information technologies including Oracle 9i/10g/11g, Oracle Fusion SOAs, and Oracle Applications.
	
Summarising up, none of the solutions in the market really facilitate end-users to build up their own applications or widgets allowing the interaction with Web services created by third-party providers. The solutions found are mainly data-oriented (RSS feeds, databases, raw text). Several tools permits some sort of (Web) service integration, but are meant to be used within an entreprise level by savvy business users or developers. [TODO: rewrite this paragraph] This paper presents an approach to leverage the possibility of integrate REST or SOAP-based web service inside a widget or any application for a web browser (by creating service wrappers), and graphical tool will be provided to facilitate the process of building these service wrappers.

In the context of publishing and discovering Web services, a service provider have well-known and widely used technologies to accomplish the task of publication, such as the Universal Description, Discovery and Integration (UDDI) or WS-Discovery. The UDDI service registry specification \cite{uddi2004} is currently one of the core standards in the Web service technology stack and an integral part of every major SOA vendor's technology strategy and offers to requesters the ability to discover services via the Internet. In a few words, UDDI serves as a centralised repository of WSDL documents. A similar concept is iServe \cite{pedrinaci_ores2010}. This platform aims to publish Web services as what they called Lined Services Linked Services --linked data describing services, storing Web service definitions in a semantic-annotation fashion so semantic Web technologies for Web service discovery and processing can make use of them. However, the platform does not deal with the step from the definition to the consumption of the services.

A different approach for Web service discovery is the Web Services Dynamic Discovery (WS-Discovery) specification. The core of this approach is a multicast discovery protocol. Service providers and consumers  listen each other for new services specifications within a network, so there is no need of a centralised registry. As a drawback, WS-Discovery do not support Internet-scale discovery, therefore make if useless for our purposes.

[TODO: rewrite/extend this] The rule based approach of the service wrapper tool has been inspired by triple graph grammars, cf. \cite{schurr94}. While general triple graph grammars allow to relate complex graph structures with each other, in our case the target of a rule is always a single object or attribute. This facilitates the whole mechanism, reasonably. 
