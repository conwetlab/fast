%!TEX root = paper.tex

\begin{abstract}
B2B systems integration was revolutionised by the introduction of web services. The way enterprise systems communicated was dramatically transformed, and by adopting these technologies, the relationships with providers and customers were strengthened. The main advantages seen by companies which adopted the technology were an increase of operational efficiencies, and reduction of costs. 
In this scenario, highly qualified software developers are responsible for the integration of services with other systems, by means of the analysis of WSDL service description or human-readable specifications. 
However this model fails when it tries to target the long tail of enterprise software demand, and as a result, the end-users. Discovery and consumption of web services is far from a straight-forward task for an end-user, meaning potential users, willing to create their task-specific applications, have been ruled out. 
This paper presents an approach to facilitate the discovery and consumption of web services by end-users on the Internet, closing the gap between business web services and the latter. The process includes: (a) a way to generate ready-to-use web services wrappers, and (b) a catalogue which users can browse to search for web services fitting their needs. These web services wrappers are described by a model created, although highly influenced by well-known web services modelling standards. Furthermore, two prototypes have been developed to serve as a proof of concept.
\keywords{web services, end-users, discovery, consumption, linked data}
\end{abstract}
