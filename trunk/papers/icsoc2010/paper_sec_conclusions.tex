
\section{Conclusions and future work}

It has been demostrated that adopting web services standards lead enterprises to increase operational efficiency, reduce costs and strength the relations with partners. Many of these standards have been widely adopted, such as WSDL, XML and UDDI, and many companies offer access to their information and operational system through web services. However, when they tackle with end-users, the process for publication and consumption is not well defined. They common way of publishing is directly by providing some sort of definition in their websites, such a WSDL document and further details in a human readable HTML page. The discovery is made through a search engine or aggregator, mainly in a syntatic-based way, and for their consumption, in most cases, high programming skills are required. The work presented in this paper empowers the end-user with a platform to easily discover services based on functional behaviour (pre/postconditions) and other metadata, being able to download a so-called resource adapter allowing the consumption of web services using languages to execute within a web browser, and a tool to transform, in an interactive manner, formal definitions of web service into these resource adapter, ready to being publish into the platform.

The current wrapping tool version permits to create resource adapters for RESTful web services. Next step is to tackle with SOAP-based web services which are described in WSDL documents, hence we will cover the two most used paradigms for leveraging web services. However, we consider to include semantically enriched WSDL documents using SAWSDL, and to support WSMO-lite services. Another limitation of the platform is inherent from web clients technologies and the cross-domain policy for security. This is solved within FAST using platform dependent API calls. Hence, the code generated makes use of the FAST API, which will be transformed depending on the mashup platform the user intends to deploy the gadget. To avoid these issues, and to be able to provide platform-independent code right away, we are researching into JSONRequest proposed by \cite{crockford2006} and other solutions.
